\documentclass[a4page]{article}
\usepackage[familydefault,light]{Chivo}
\usepackage[luatex]{graphicx}
\usepackage[none]{hyphenat}
\usepackage[english]{babel}
%\usepackage[hyphens]{url}
%\usepackage{fancyhdr}
%\usepackage{titlesec}
\usepackage{hyperref}
%
\newfontfamily{\emojiFont}{Noto Color Emoji}[Renderer=Harfbuzz]
%\usepackage{emji}
%
\title{Taxonomy of Crosses}
\author{Tomasz Przechlewski}
%%
\begin{document}
\maketitle
\rightskip0pt plus 3em

\section{Shape variants}

Basically as 90\% population in Poland declares themselves as Roman
catholic we (and Protestants ie Christians who departed from
Catholicism) use Roman variant of the cross and I am accustomed to this
variant.  At the east (which was once a part of Russian Empire) lives
orthodox minority and they use \emph{Suppedaneum\/} cross
(\url{https://en.wikipedia.org/wiki/Russian_Orthodox_cross}) which is
Russian Orthodox cross not Orthodox cross as I falsely thought (or not
as Unicode claims it is Orthodox cross--see below).

What is `proper' an orthodox cross is a mystery to me.
Perhaps there is not such a thing. But one of oldest cross shapes is

\url{https://en.wikipedia.org/wiki/Patriarchal_cross}

developed in Byzantine Empire a~long time ago--so definitely it is 
is an orthodox one... This shape is also known
as \strong{Anjou} or \strong{Lorraine}.
It is just cross with two bars.

The whole story about Cross of Lorraine:

\url{https://frenchmoments.eu/cross-of-lorraine/}

(note some connections to Jordan:-))

Or \url{https://en.wikipedia.org/wiki/Cross_of_Lorraine}

Note that it originating on the East, then become popular in France
and finally it was used in the whole Europe (as it can be found 
on coat of arms in Lithuania, Hungary
and Solvenia (see Wiki page). -- all countries are historically catholic.)

\section{Production variants}

The other issue is the material used. Usually stone or metal (gold:-)) but
also wood. Wooden crosses especially made of birch are associated in Poland
with fallen war heroes/soldiers. They were buried quickly 
and they graves were marked with crosses made of something readily available at the
battlefield. Most of them disappeared for obvious reason, 
but some were preserved in its original form:

\url{https://pl.m.wikipedia.org/wiki/Plik:Kwatera_Batalionu_Zo%C5%9Bka_Cmentarz_Wojskowy_na_Pow%C4%85zkach_010.JPG}

Another interesting example is a~cross made from cast-iron. Popular in XIX century.
Basically iron crossed were casted or forged. In both cases by local craftsmen.

\texttt{google:cast+iron+cross (click images)} 

Very original are crosses (and paintings) developed in Ethiopia (orthodox church):

\url{https://en.wikipedia.org/wiki/Ethiopian_cross}

\url{https://en.wikipedia.org/wiki/Ethiopian_Orthodox_Tewahedo_Church}

see also:

\texttt{google:ethiopia+icon+church (click images)}

Paintings (icons) are painted on the on goat skin.


\section{Unicode and crosses}

A~few crosses were defined in Unicode and 
developed in various fonts including popular
Emoji font (for example Google Noto \url{https://www.google.com/get/noto/help/emoji/}):

\url{https://unicode-table.com/en/search/?q=cross}

\strong{U2626} ({\emojiFont\char"2626}) is orthodox cross with quite a long description
about the meaning of the symbol

\url{https://unicode-table.com/en/2626/}

Unfortunately for Lorraine cross (\strong{U2628} or {\emojiFont\char"2628})
the description is much shorter:

\url{https://unicode-table.com/en/2628/}

See also:

\url{https://emojipedia.org/symbols/}

\url{https://fsymbols.com/signs/cross/}


\end{document}
% Local Variables:
% TeX-master: ""
% mode: latex
% coding: utf-8
% ispell-local-dictionary: "english"
% End:
